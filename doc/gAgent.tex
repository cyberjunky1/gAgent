\documentclass[a4paper]{article}

\usepackage[french]{babel}
\usepackage[utf8x]{inputenc}
\usepackage{float}
\usepackage{amsmath}
\usepackage{graphicx}
\usepackage[colorinlistoftodos]{todonotes}
\usepackage{url}
\usepackage{array}
\usepackage{tabularx}
%pour utiliser \floatbarrier
%\usepackage{placeins}
%\usepackage{floatrow}
%espacement entre les lignes
\usepackage{setspace}
\usepackage[hypertexnames=false, pdftex]{hyperref}
\hypersetup{ colorlinks = true, linkcolor = black, urlcolor = blue, citecolor = blue }
%\usepackage[T1]{fontenc}
\usepackage[top=2cm, bottom=2cm, left=2cm, right=2cm]{geometry}
%Pour les galerie d'images
\usepackage{subfig}


\usepackage{listings}
\usepackage{xcolor}
\lstset { %
    language=C++,
    backgroundcolor=\color{blue!5}, % set backgroundcolor
    basicstyle=\footnotesize,% basic font setting
    %basicstyle=\ttfamily,
    keywordstyle=\color{blue}\ttfamily,
    stringstyle=\color{red}\ttfamily,
    commentstyle=\color{green}\ttfamily,
    morecomment=[l][\color{magenta}]{\#},
    showlines=true
}



%======================== DEBUT DU DOCUMENT ========================

\begin{document}

%régler l'espacement entre les lignes
\newcommand{\HRule}{\rule{\linewidth}{0.5mm}}

%page de garde
\begin{titlepage}
\begin{center}

% Upper part of the page. The '~' is needed because only works if a paragraph has started.
% \includegraphics[width=0.35\textwidth]{./logo}~\\[1cm]

\textsc{\LARGE Université Paris8}\\[1.5cm]

\textsc{\Large }\\[0.5cm]

% Title
\HRule \\[0.4cm]

{\huge \bfseries gAgent \\
Framework multi-agent en C++ \\[0.4cm] }

\HRule \\[1.5cm]

% Author and supervisor
\begin{minipage}{0.4\textwidth}
\begin{flushleft} \large
\emph{Auteur:}\\
Halim Djerroud \textit{hdd@ai.univ-paris8.fr}\\
\end{flushleft}
\end{minipage}


\vfill

% Bottom of the page
{\large \today}

\end{center}
\end{titlepage}





\tableofcontents
\thispagestyle{empty}
\setcounter{page}{0}
%ne pas numéroter le sommaire

\newpage

%espacement entre les lignes d'un tableau
\renewcommand{\arraystretch}{1.5}

%====================== INCLUSION DES PARTIES ======================

\section{Introduction}
\textbf{gAgent} est une framework écrit en C++ qui permet de développer un système multi-agent compatible FIPA\footnote{\url{http://www.fipa.org}} en C++ et en python.


La spécification FIPA


\section{Architecture gAgent}

Dans gAgent, chaque agent est un processus linux. Chaque agent (processus) est composé d'au moins deux threads :
\begin{itemize}
 \item Listner :  Ce thread est un thread détaché , il est implémenté sous forme de service, il permet de gérer les événements externes , 
		  des signaux Linux de \texttt{SIGRTMIN + 2} à \texttt{SIGRTMIN + 7}.
 \item Controler : Ce thread est un thread détaché aussi, il permet de gérer les verrous.
\end{itemize}


\section{Premier pas}
Le plus petit programme gAgent:

\begin{lstlisting}

	#include <gagent/AgentCore.hpp>
	using namespace gagent;
	int main(int argc, char* argv[]) {
		AgentCore::initAgentSystem();
		...
		AgentCore::stopAgentSystem();
		return EXIT_SUCCESS;
	}
	
\end{lstlisting}

\texttt{\#include <gagent\/AgentCore.hpp>}: Inclure le moteur principale de gAgent.

\texttt{AgentCore::initAgentSystem()}: Permet d'initialiser \textbf{gAgent} (lire le fichier config).

\texttt{AgentCore::stopAgentSystem()}: Permet de libérer les ressources \textbf{gAgent}.

\section{Créer un premier Agent}

Les agents \textbf{gAgent} peuvent être dans d'un de ces états :

\begin{figure}[thpb]
  \centering
    \includegraphics[scale=0.475]{./img/cycle.png}
    \caption{Cycle de vie d'un agent gAgent.}
    \label{cycle_de_vie} 
\end{figure}


\begin{itemize}
 \item Setup: À ce stade l'agent existe en mémoire mais sont état est inconnu (UNKNOWN). Pour activer l'agent il faut appeler la méthode \texttt{doInit()}
 \item Suspendu: La méthode \texttt{doSuspend()} permet de suspendre un agent qui est dans l'état Active, dans ce nouveau état (Suspendu) 
		 l'agent arrête tout exécution. Seule la méthode \texttt{doActivate()} permet le faire sortir de cet état pour revenir dans l'état active.
 \item En attente: La méthode \texttt{doWait()} permet de mettre l'agent en état d'attente d'un événement (généralement un message) puis il revient dans sont état active lors de la réception
		 de l'événement ou-bien on peut le forcer son retour dans l'état active avec la méthode \texttt{doWake()}
 \item Active: Dans cet état l'agent est en cours d'exécution.
 \item Transit: La méthode \texttt{doMove()} permet la migration des agents d'une plateforme à une autre.
 \item Fin: La méthode \texttt{doDelete()} permet d'arrêter d'un agent en exécution et libérer les ressources.
\end{itemize}



\begin{lstlisting}

	#include <gagent/Agent.hpp>
	#include <gagent/AgentCore.hpp>
	using namespace gagent;

	class myAgent: public Agent {
	public:
	  myAgent() :Agent() {};
	  virtual ~myAgent() {}
	  void setup() {
	    ...
	  }
	};
	
	int main(int argc, char* argv[]) {
	    AgentCore::initAgentSystem();
	    myAgent* g = new myAgent();
	    
	    g->init();
	    
	    sleep(3);
	    g->doWait();
	    
	    sleep(3);
	    g->doActivate();
	    
	    sleep(3);
	    g->doMove();
	    
	    sleep(3);
	    g->doSuspend();
	    
	    sleep(3);
	    g->doActivate();
	    
	    sleep(3);
	    g->doWait();
	    
	    sleep(3);
	    g->doWake();
	    
	    sleep(3);
	    g->doDelete();
	    
	    delete g;
	    
	    AgentCore::stopAgentSystem();
	    return EXIT_SUCCESS;
	}
	
\end{lstlisting}

Les méthodes : 

\begin{itemize}
 \item \texttt{setup()} : invoqué lors de l'initialisation de l'agent.
 \item \texttt{takedown()} :  invoquée avant qu’un agent ne quitte la plateforme.
\end{itemize}



\section{Les comportements d'un agent gAgent}

Pour qu'un agent exécute une tâche, ou plusieurs, if faut auparavant définir ces tâches. Les tâches sont appelées \textbf{behaviours} (ou comportements en français)
sont des instances de la classe \textbf{Behaviour} ou une des ses sous classes.

Pour qu'un agent exécute une tâche il doit lui l'attribuer un ou plusieurs \textbf{Behaviour} par la méthode \texttt{addBehaviour(Behaviour b)} de la classe \textbf{Agent} (voir l'exemple qui suit).

\textbf{gAgent} implémente chaque \textbf{Behaviour} dans un thread, si plusieurs \textbf{Behaviour} sont implémenté par l'agent, alors ces \textbf{Behaviours} sont exécutés en parallèles.

Chaque \textbf{Behaviour} doit obligatoirement implémenter les deux méthodes suivantes :

\begin{itemize}
 \item \texttt{action()} : l'utilisateur doit implémenter les opérations à exécuter par le \textbf{Behaviour};
 \item \texttt{done()} : qui indique si le \textbf{Behaviour} en question a terminé son exécution ou pas.
\end{itemize}


Les deux autres méthodes suivantes, dont l'implémentation n'est pas obligatoire mais qui peuvent s'avérer utiles :

\begin{itemize}
 \item \texttt{onStart()} : appelée juste avant l'exécution de ma méthode \texttt{action()};
 \item \texttt{onEnd()} : appelée juste après la retournement de \textit{true} par la méthode \texttt{done()}.
\end{itemize}

Si besoin de savoir quel est le propriétaire d'un \textbf{Behaviour}, et cela peut être connu par le membre \texttt{this\_agent} du \textbf{Behaviour} en question. 

Exemple d'implémentation d'un Behaviour:

\begin{lstlisting}

	#include <iostream>
	#include <stdio.h>

	#include <gagent/Behaviour.hpp>
	#include <gagent/Agent.hpp>
	#include <gagent/AgentCore.hpp>

	using namespace gagent;

	class myCycle: public CyclicBehaviour {
	public:
		myCycle(Agent* ag) :CyclicBehaviour(ag) {
		}

		void action() {
			std::cout << "Coin " << std::flush;
			sleep(1);
		}
	};

	class myAgent: public Agent {
	public:
		myAgent() :Agent() {};

		virtual ~myAgent() {}

		void setup() {
			myCycle* bb = new myCycle(this);
			addBehaviour(bb);
		}
	};

	int main() {

		AgentCore::initAgentSystem();
		myAgent* g = new myAgent();
		g->init();

		AgentCore::syncAgentSystem();
		AgentCore::stopAgentSystem();

		return 0;
	}
	
\end{lstlisting}

Résultat : 

\begin{lstlisting}[backgroundcolor=\color{green!5}]

	Coin Coin Coin Coin Coin Coin Coin 
 
\end{lstlisting}


gAgent propose les Behaviours suivants: 

\begin{itemize}
 \item OneShotBehaviour
 \item SimpleBehaviour
 \item CompositeBehaviour
 \item CyclicBehaviour
 \item WakerBehaviour
 \item ParallelBehaviour
 \item SequentialBehaviour
 \item FSMBehaviour
\end{itemize}

\subsection{Les Behaviours simples}

C'est un Behaviour qui donner au 

Ils sont composés de  Behaviours : OneShotBehaviour, CyclicBehaviour.

\begin{lstlisting}
	...
	class mySimpleBehavior: public SimpleBehaviour {
	public:
		mySimpleBehavior(Agent* ag) : SimpleBehaviour(ag) {}

		unsigned int i = 0;

		void onStart(){
			std::cout << "Je sais compter de 1 a 10 " 
				  << std::endl << std::flush;
			i=0;
		}

		void action() {
			i++;
			std::cout <<  i  << std::endl << std::flush;
		}

		bool done(){
			if (i==10){
				std::cout << "J ai fini de compter :) ." 
					  << std::endl << std::flush;
				return true;
			}
			return false;
		}
	};

	class myAgent: public Agent {
	public:
		myAgent() :Agent() {};

		virtual ~myAgent() {}

		void setup() {
			mySimpleBehavior* b1 = new mySimpleBehavior(this);
			addBehaviour(b1);

		}
	};

\end{lstlisting}



Résultat : 

\begin{lstlisting}[backgroundcolor=\color{green!5}]

	Je sais compter de 1 a 10 
	1
	2
	3
	4
	5
	6
	7
	8
	9
	10
	J ai fini de compter :) .

\end{lstlisting}


\subsubsection{OneShotBehaviour}

Le Behaviour \textbf{OneShotBehaviour} permet d'exécuter une tâche une et une seule fois puis il se termine. La classe 
OneShotBehaviour implémente la méthode \texttt{done()} et elle retourne toujours \textit{true}.

\begin{lstlisting}
	...
	using namespace gagent;

	class myBehavior: public OneShotBehaviour {
	public:
		myBehavior(Agent* ag) : OneShotBehaviour(ag) {
		}

		void onStart() {
			std::cout << " -- start -- " << std::endl << std::flush;
		}

		void action() {
			std::cout << "Coin " << std::endl << std::flush;
		}

		void onEnd() {
			std::cout << " -- end -- "   << std::endl << std::flush;
		}
	};

	class myAgent: public Agent {
	public:
		...
		void setup() {
			myBehavior* bb = new myBehavior(this);
			addBehaviour(bb);
		}
	};

	int main() {
		...
		myAgent* g = new myAgent();
		g->init();
		...
	}
\end{lstlisting}

Résultat : 

\begin{lstlisting}[backgroundcolor=\color{green!5}]

	-- start -- 
	Coin 
	-- end -- 

\end{lstlisting}

Un exemple avec deux Behaviours : 

\begin{lstlisting}

	  ...
	  class myBehavior1: public OneShotBehaviour {
	  public:
		  myBehavior1(Agent* ag) : OneShotBehaviour(ag) {}
		  void action() {
			  std::cout << "Comportement 1 " << std::endl << std::flush;
		  }
	  };

	  class myBehavior2: public OneShotBehaviour {
	  public:
		  myBehavior2(Agent* ag) : OneShotBehaviour(ag) {}
		  void action() {
			  std::cout << "Comportement 2 " << std::endl << std::flush;
		  }
	  };

	  class myAgent: public Agent {
	  public:
		 ...
		  void setup() {
			  myBehavior1* b1 = new myBehavior1(this);
			  addBehaviour(b1);

			  myBehavior2* b2 = new myBehavior2(this);
			  addBehaviour(b2);
		  }
	  };
	  
	  int main() {...
	  
\end{lstlisting}

Résultat:

\begin{lstlisting}[backgroundcolor=\color{green!5}]

	Comportement 1 
	Comportement 2 

\end{lstlisting}



\subsubsection{Cyclic Behaviour}

Le \textbf{Cyclic Behaviour} permet d'exécuter la méthode \textbf{action()} de façon continue indéfiniment. 

\begin{lstlisting}
	...
	class myCycle: public CyclicBehaviour {
	public:
		myCycle(Agent* ag) :CyclicBehaviour(ag) {
		}

		void action() {
			std::cout << "Coin " << std::flush;
			sleep(1);
		}
	};

	class myAgent: public Agent {
	public:
		myAgent() :Agent() {};

		virtual ~myAgent() {}

		void setup() {
			myCycle* bb = new myCycle(this);
			addBehaviour(bb);
		}
	};

	int main() {

		AgentCore::initAgentSystem();
		myAgent* g = new myAgent();
		g->init();
		AgentCore::syncAgentSystem();
		AgentCore::stopAgentSystem();
		return 0;
	}
	
\end{lstlisting}

Résultat:

\begin{lstlisting}[backgroundcolor=\color{green!5}]

	Coin Coin Coin Coin Coin Coin Coin Coin Coin Coin ...

\end{lstlisting}



\subsection{Les Behaviours planifiés}

\subsubsection{WakerBehaviour}

Le \textbf{WakerBehaviour} permet d'exécuter la méthode \texttt{onWake()} après un délai passé en paramètre en millisecondes.

\begin{lstlisting}

	...
	using namespace gagent;

	class myWakerBehaviour: public WakerBehaviour {
	public:
		myWakerBehaviour(Agent* ag) : WakerBehaviour(ag,50000) { }

		void onWake(){
			std::cout << "Termine " << std::endl << std::flush;
		}
	};

	class myAgent: public Agent {
	public:
		myAgent() :
				Agent() {
		};

		virtual ~myAgent() {
		}

		void setup() {
			myWakerBehaviour* b = new myWakerBehaviour(this);
			addBehaviour(b);
		}
	};

	int main() {
		...
		g->init();
		...
	}

\end{lstlisting}

Résultat:

\begin{lstlisting}[backgroundcolor=\color{green!5}]
	
	//apres une attente de 5 secondes.
	Termine

\end{lstlisting}

\subsubsection{TickerBehaviour}

Le \textbf{TickerBehaviour} est implémenté pour qu'il exécute sa tâche périodiquement par la méthode onTick(). La durée de la période est passée comme argument au constructeur en millisecondes. 

\begin{lstlisting}

	...
	class myWakerBehaviour: public WakerBehaviour {
	public:
		myWakerBehaviour(Agent* ag) : WakerBehaviour(ag,5000) { }

		void onWake(){
			std::cout << " Termine " << std::endl << std::flush;
			this_agent->doDelete();
		}

	};

	class myTickerBehaviour: public TickerBehaviour {
	public:
		myTickerBehaviour(Agent* ag) : TickerBehaviour(ag,1000) { }

		void onTick(){
			std::cout << " tictac " << std::endl << std::flush;
		}
	};
	
	class myAgent: public Agent {
	public:
		myAgent() :
			Agent() {
		};
		void setup() {
			myWakerBehaviour* b3 = new myWakerBehaviour(this);
			addBehaviour(b3);
			myTickerBehaviour* b4 = new myTickerBehaviour(this);
			addBehaviour(b4);
		}
	};

	int main() {

		AgentCore::initAgentSystem();
		myAgent* g = new myAgent();
		g->init();

		AgentCore::syncAgentSystem();
		AgentCore::stopAgentSystem();
		return 0;
	}

\end{lstlisting}

Résultat : 

\begin{lstlisting}[backgroundcolor=\color{green!5}]
	
	tictac 
	tictac 
	tictac 
	tictac 
	tictac 
	Termine

\end{lstlisting}

\section{La communication entre les agents}



\section{Le plateforme gAgent}



\section{Monitoring}

\textbf{gAgent} propose un outil de monitoring sur une console. pour lancer le monitoring il faut appeler le programme \texttt{agentmonitor}. 
C'est un service qui permet de recevoir des messages des agents en utilisant le protocole UDP.

L'adresse IP et le PORT sont définis en paramètres sinon le programme contente de lire le fichier de configuration \texttt{config.cfg} dans lequel sont ces paramètres sont définis.

\begin{lstlisting}[backgroundcolor=\color{green!5}]

	$agentmonitor --help
	Allowed options:
	--help                produce help message
	--port arg            port
	--ip arg              Ip adress

\end{lstlisting}


On peut tester cet outil avec la commande Unix \texttt{nc} comme suite \texttt{echo 'hello' | nc -u <ip> <port>}.

exemple : 

\texttt{echo 'hello' | nc -u 127.0.0.1 40013}


\begin{lstlisting}[backgroundcolor=\color{green!5}]
	
	000001 : hello

\end{lstlisting}

L'objectif de cet outil est permettre à l'utilisateur de surveiller le système durant le son fonctionnement.

Les agent peuvent envoyer les messages (en C++) sur le monitor comme suite : 


\begin{lstlisting}

	this_agent->sendMsgMonitor(" Hello ");

\end{lstlisting}

Résultat :

\begin{lstlisting}[backgroundcolor=\color{green!5}]

	000001 : EbcrjZlp -> Start agent PID : 13029 
	000002 : EbcrjZlp ->  tictac  
	000003 : EbcrjZlp ->  tictac  
	000004 : EbcrjZlp ->  tictac  
	000005 : EbcrjZlp ->  tictac  
	000006 : EbcrjZlp ->  tictac  
	000007 : EbcrjZlp -> Stop agent PID : 13029 
  
\end{lstlisting}



%récupérer les citation avec "/footnotemark"
\nocite{*}

%choix du style de la biblio
\bibliographystyle{plain}
%inclusion de la biblio
%\bibliography{bibliographie.bib}
%voir wiki pour plus d'information sur la syntaxe des entrées d'une bibliographie

\end{document}